% Options for packages loaded elsewhere
\PassOptionsToPackage{unicode}{hyperref}
\PassOptionsToPackage{hyphens}{url}
%
\documentclass[
]{article}
\usepackage{lmodern}
\usepackage{amssymb,amsmath}
\usepackage{ifxetex,ifluatex}
\ifnum 0\ifxetex 1\fi\ifluatex 1\fi=0 % if pdftex
  \usepackage[T1]{fontenc}
  \usepackage[utf8]{inputenc}
  \usepackage{textcomp} % provide euro and other symbols
\else % if luatex or xetex
  \usepackage{unicode-math}
  \defaultfontfeatures{Scale=MatchLowercase}
  \defaultfontfeatures[\rmfamily]{Ligatures=TeX,Scale=1}
\fi
% Use upquote if available, for straight quotes in verbatim environments
\IfFileExists{upquote.sty}{\usepackage{upquote}}{}
\IfFileExists{microtype.sty}{% use microtype if available
  \usepackage[]{microtype}
  \UseMicrotypeSet[protrusion]{basicmath} % disable protrusion for tt fonts
}{}
\makeatletter
\@ifundefined{KOMAClassName}{% if non-KOMA class
  \IfFileExists{parskip.sty}{%
    \usepackage{parskip}
  }{% else
    \setlength{\parindent}{0pt}
    \setlength{\parskip}{6pt plus 2pt minus 1pt}}
}{% if KOMA class
  \KOMAoptions{parskip=half}}
\makeatother
\usepackage{xcolor}
\IfFileExists{xurl.sty}{\usepackage{xurl}}{} % add URL line breaks if available
\IfFileExists{bookmark.sty}{\usepackage{bookmark}}{\usepackage{hyperref}}
\hypersetup{
  pdftitle={r2redux},
  pdfauthor={Moksedul Momin and Hong Lee},
  hidelinks,
  pdfcreator={LaTeX via pandoc}}
\urlstyle{same} % disable monospaced font for URLs
\usepackage[margin=1in]{geometry}
\usepackage{graphicx,grffile}
\makeatletter
\def\maxwidth{\ifdim\Gin@nat@width>\linewidth\linewidth\else\Gin@nat@width\fi}
\def\maxheight{\ifdim\Gin@nat@height>\textheight\textheight\else\Gin@nat@height\fi}
\makeatother
% Scale images if necessary, so that they will not overflow the page
% margins by default, and it is still possible to overwrite the defaults
% using explicit options in \includegraphics[width, height, ...]{}
\setkeys{Gin}{width=\maxwidth,height=\maxheight,keepaspectratio}
% Set default figure placement to htbp
\makeatletter
\def\fps@figure{htbp}
\makeatother
\setlength{\emergencystretch}{3em} % prevent overfull lines
\providecommand{\tightlist}{%
  \setlength{\itemsep}{0pt}\setlength{\parskip}{0pt}}
\setcounter{secnumdepth}{-\maxdimen} % remove section numbering

\title{r2redux}
\author{Moksedul Momin and Hong Lee}
\date{10/08/2022}

\begin{document}
\maketitle

\hypertarget{r2redux}{%
\section{r2redux}\label{r2redux}}

The `r2redux' package can be used to derive test statistics for R\^{}2
values from polygenic risk score (PGS) models (variance and covariance
of R\^{}2 values, p-value and 95\% confidence intervals (CI)). For
example, it can test if two sets of R\^{}2 values from two different PGS
models are significantly different to each other whether the two sets of
PGS are independent or dependent. Because R\^{}2 value is often regarded
as the predictive ability of PGS, r2redux package can be useful to
assess the performances of PGS methods or multiple sets of PGS based on
different information sources. Furthermore, the package can derive the
information matrix of β ̂\_1\^{}2and β ̂\_2\^{}2 from a multiple
regression (see olkin\_beta1\_2 or olkin\_beta\_info function in the
manual), which is a basis of a novel PGS-based genomic partitioning
method (see r2\_enrich or r2\_enrich\_beta function in the manual). It
is recommended that the target sample size in the PGS study should be
more than 2,000 for quantitative traits and more than 5,000 for binary
responses or case-control studies. The P-value generated from the
r2redux package provides two types of p-values (for one- and two-tailed
test) unless the comparison is for nested models (e.g.~y=PGS\_1+PGS\_2+e
vs.~y=PGS\_2+e) where the R\^{}2 of the full model is expected to be
always higher than the reduced model. When there are multiple covariates
(e.g.~age, sex and other demographic variables), the phenotypes can be
adjusted for the covariates, and pre-adjusted phenotypes (residuals)
should be used in the r2redux.

\hypertarget{installation}{%
\section{INSTALLATION}\label{installation}}

To use r2redux:

\begin{verbatim}
install.packages("r2redux") 
library(r2redux)
\end{verbatim}

or

\begin{verbatim}
install.packages("devtools")
library(devtools)
devtools::install_github("mommy003/r2redux")
library(r2redux)
\end{verbatim}

\hypertarget{quick-start}{%
\section{QUICK START}\label{quick-start}}

We illustrate the usage of r2redux using multiple sets of PGS estimated
based on GWAS summary statistics from UK Biobank or Biobank Japan
(reference datasets). In a target dataset, the phenotypes of target
samples (y) can be predicted with PGS (a PGS model, e.g.~y=PRS+e, where
y and PGS are column-standardised 1. Note that the target individuals
should be independent from reference individuals. We can test the
significant differences of the predictive ability (R\^{}2) between a
pair of PGS (see r2\_diff function and example in the manual).

\hypertarget{data-preparation}{%
\section{DATA PREPARATION}\label{data-preparation}}

\textbf{a. Statistical testing of significant difference between R2
values for p-value thresholds:} r2redux requires only phenotype and
estimated PGS (from PLINK or any other software). Note that any missing
value in the phenotypes and PGS tested in the model should be removed.
If we want to test the significant difference of R\^{}2 values for
p-value thresholds, r2\_diff function can be used with an input file
that includes the following fields (also see
test\_ukbb\_thresholds\_scaled in the example directory form github
(\url{https://github.com/mommy003/r2redux}) or read dat1 file embedded
within the package and r2\_diff function in the manual).

\begin{itemize}
\tightlist
\item
  Phenotype (y)
\item
  PGS for p value 1 (x1)
\item
  PGS for p value 0.5 (x2)
\item
  PGS for p value 0.4 (x3)
\item
  PGS for p value 0.3 (x4)
\item
  PGS for p value 0.2 (x5)
\item
  PGS for p value 0.1 (x6)
\item
  PGS for p value 0.05 (x7)
\item
  PGS for p value 0.01 (x8)
\item
  PGS for p value 0.001 (x9)
\item
  PGS for p value 0.0001 (x10)
\end{itemize}

To get the test statistics for the difference between R2(y=x{[},v1{]})
and R2(yx{[},v2{]}). (here we define R\_1\^{}2= R\^{}2(y=x{[},v1{]}))
and R\_2\textsuperscript{2=R}2(y=x{[},v2{]})))

\begin{verbatim}
dat=read.table("test_ukbb_thresholds_scaled") #(see example files) or
dat=dat1 #(this example embedded within the package)
nv=length(dat$V1)
v1=c(1)
v2=c(2)
output=r2_diff(dat,v1,v2,nv)
\end{verbatim}

\begin{itemize}
\tightlist
\item
  r2redux output
\item
  output\$var1 (variance of R\_1\^{}2)
\item
  0.0001437583
\item
  output\$var2 (variance of R\_2\^{}2)
\item
  0.0001452828
\item
  output\$var\_diff (variance of difference between R\_1\^{}2and
  R\_2\^{}2)
\item
  5.678517e-07
\item
  output\$r2\_based\_p (p-value for significant difference between
  R\_1\^{}2 and R\_2\^{}2)
\item
  0.5514562
\item
  output\$mean\_diff (differences between R\_1\^{}2 and R\_2\^{}2)
\item
  -0.0004488044
\item
  output\$upper\_diff (upper limit of 95\% CI for the difference)
\item
  0.001028172
\item
  output\$lower\_diff (lower limit of 95\% CI for the difference)
\item
  -0.001925781
\end{itemize}

\textbf{b. PGS-based genomic enrichment analysis:} If we want to perform
some enrichment analysis (e.g., regulatory vs non\_regulatory) in the
PGS context to test significantly different from the expectation
(p\_exp= \# SNPs in the regulatory / total \# SNPs = 4\%). We
simultaneously fit two sets of PGS from regulatory and non-regulatory to
get β ̂\_regu\^{}2 and β ̂\_(non-regu)\^{}2, using a multiple regression,
and assess if the ratio, (β ̂\_1\^{}2)/(r\_(y,〖(x〗\_1,x\_2))\^{}2 ) are
significantly different from the expectation, p\_exp. To test this, we
need to prepare input file for r2redux that includes the following
fields (e.g.~test\_ukbb\_enrichment\_choles in example directory or read
dat2 file embedded within the package and r2\_enrich\_beta function in
the manual).

\begin{itemize}
\tightlist
\item
  Phenotype (y)
\item
  PGS for regulatory region (x1)
\item
  PGS for non-regulatory region (x2)
\end{itemize}

To get the test statistic for the ratio which is significantly different
from the expectation. var(β ̂\_1\textsuperscript{2/r\_(y,(x\_1,x\_2))}2),
where β ̂\_1\^{}2 is the squared regression coefficient of x\_1 from a
multiple regression model, i.e.~y=x\_1 β\_1+ x\_2 β\_2+e, and
r\_(y,(x\_1,x\_2))\^{}2 is the coefficient of determination of the
model. It is noted that y, x\_1 and x\_2 are column standardised (mean 0
and variance 1).

\begin{verbatim}
dat=read.table("test_ukbb_enrichment_choles") #(see example file) or 
dat=dat2 #(this example embedded within the package)
nv=length(dat$V1)
v1=c(1)
v2=c(2)
dat=dat2
nv=length(dat$V1)
v1=c(1)
v2=c(2)
output=r2_beta_var(dat,v1,v2,nv)
\end{verbatim}

\begin{itemize}
\tightlist
\item
  r2redux output
\item
  output\$beta1\_sq (beta1\^{}2)
\item
  0.01118301
\item
  output\$beta2\_sq (beta2\^{}2)
\item
  0.004980285
\item
  output\$var1 (variance of beta1\^{}2)
\item
  7.072931e-05
\item
  output\$var2 (variance of beta2\^{}2)
\item
  3.161929e-05
\item
  output\$var1\_2 (variance of difference between beta1\^{}2 and
  beta2\^{}2)
\item
  0.000162113
\item
  output\$cov (covariance between beta1\^{}2 and beta2\^{}2)
\item
  -2.988221e-05
\item
  output\$upper\_beta1\_sq (upper limit of 95\% CI for beta1\^{}2)
\item
  0.03037793
\item
  output\$lower\_beta1\_sq (lower limit of 95\% CI for beta1\^{}2)
\item
  -0.00123582
\item
  output\$upper\_beta2\_sq (upper limit of 95\% CI for beta2\^{}2)
\item
  0.02490076
\item
  output\$lower\_beta2\_sq (lower limit of 95\% CI for beta2\^{}2)
\item
  -0.005127546
\end{itemize}

\begin{verbatim}
dat=dat2 #(this example embedded within the package)
nv=length(dat$V1)
v1=c(1)
v2=c(2)
expected_ratio=0.04
output=r2_enrich_beta(dat,v1,v2,nv,expected_ratio)
\end{verbatim}

\begin{itemize}
\tightlist
\item
  r2redux output
\item
  output\$beta1\_sq (beta1\^{}2)
\item
  0.01118301
\item
  output\$beta2\_sq (beta2\^{}2)
\item
  0.004980285
\item
  output\$ratio1 (beta1\textsuperscript{2/R}2)
\item
  0.4392572
\item
  output\$ratio2 (beta2\textsuperscript{2/R}2)
\item
  0.1956205
\item
  output\$ratio\_var1 (variance of ratio 1)
\item
  0.08042288
\item
  output\$ratio\_var2 (variance of ratio 2)
\item
  0.0431134
\item
  output\$upper\_ratio1 (upper limit of 95\% CI for ratio 1)
\item
  0.9950922
\item
  output\$lower\_ratio1 (lower limit of 95\% CI for ratio 1)
\item
  -0.1165778
\item
  output\$upper\_ratio2 upper limit of 95\% CI for ratio 2)
\item
  0.6025904
\item
  output\$lower\_ratio2 (lower limit of 95\% CI for ratio 2)
\item
  -0.2113493
\item
  output\$enrich\_p1 (two tailed P-value for beta1\textsuperscript{2/R}2
  is significantly different from exp1)
\item
  0.1591692
\item
  output\$enrich\_p1\_one\_tail (one tailed P-value for
  beta1\textsuperscript{2/R}2 is significantly different from exp1)
\item
  0.07958459
\item
  output\$enrich\_p2 (P-value for beta2\^{}2/R2 is significantly
  different from (1-exp1))
\item
  0.000232035
\item
  output\$enrich\_p2\_one\_tail (one tailed P-value for beta2\^{}2/R2 is
  significantly different from (1-exp1))
\item
  0.0001160175
\end{itemize}

\hypertarget{references}{%
\section{References}\label{references}}

\begin{enumerate}
\def\labelenumi{\arabic{enumi}.}
\tightlist
\item
  Olkin, I. and Finn, J.D. Correlations redux. Psychological Bulletin,
  1995. 118(1): p.~155.
\item
  Momin, M.M., Lee, S., Wray, N.R. and Lee S.h. 2022. Significance tests
  for R2 of out-of-sample prediction using polygenic scores. bioRxiv.
\end{enumerate}

\hypertarget{contact-information}{%
\section{Contact information}\label{contact-information}}

Please contact Hong Lee
(\href{mailto:hong.lee@unisa.edu.au}{\nolinkurl{hong.lee@unisa.edu.au}})
or Moksedul Momin
(\href{mailto:cvasu.momin@gmail.com}{\nolinkurl{cvasu.momin@gmail.com}})
if you have any queries.

\end{document}
